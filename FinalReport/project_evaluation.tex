\newpage
\chapter{Project Evaluation}
\label{chap:evaluation}
% provide feedback about the project course: what was good/fun/etc.?, what did you learn?, what didn't work well?, what can be improved for future projects?, other comments...
%This is mainly for feedback to Thomas, Kjell and LTU.

\section{Overall Project Goals and Wrap-up}



Final budget status

What project goals were achieved?

Forward propulsion and steering and battery powered flight achieved along with feasibility of using solar power. No solar powered flight yet. Capability of system to support small scientific payloads (data handling, power supply, mechanical mounting etc.) No outdoors flight yet.

Verification of possibility to design a light-weight system with maximum weight 2.5 kg (including solar cells) which can be lifted by the Esrange blimp.

\section{Pedro}


\section{Omair}


\section{Jan}


\section{Morten}
In overall, the project course was personally for me a useful experience. As a quality/project manager, I experienced many of the typical challenges in project management such as time delays, rephrasing of project goals and system concepts as required, bureaucracy issues, budget constraints, changes in team structure (people leaving, people joining), challenges getting access to components and equipment/labs etc. However, it is also my personal belief that many of these problems this project encountered could easily have been avoided with some simple planning and organization beforehand. This would allow a much more advanced technical system to be developed, putting more focus on hardcore real-life situation problems and probably leading to a greater success for the project and provide a source for good PR for LTU and the students when the interesting project results are shown in media. It is my belief that LTU has some great facilities to support very interesting student projects and they should thrive to exploit and nurture this opportunity.

Some suggestions/comments for future projects:
Establishment of Viking room to be permanent project course room. 
Permanent establishment or access to "professional" solder station, possibly even with ESD-protected table + wrist bands.
Throughout the project, we were extremely dependent on Lars, e.g. to get simple components/tools such as resistors, wires etc. which cost us a lot of time, whenever he was not available. - if possible, a box of simple components (resistors, capacitors, simple ICs, wires etc.) should be made available to at least one person (if not all) in the team. A simple account-sheet could be placed next to the box, where people can note what they take.
Quicker release of initial project budget to allow first order of components.  
Access to workshop, once established, and help from Lars worked very well. This access should be established in the very beginning of the project.
Generally we received good support from IRF with borrowing tools and suggestions for mechanical design. If possible, more instructions and recommendations from IRF experts would be helpful (soldering course, how to check solderings under microscope).
More critical design feedback and suggestions from supervisors, where possible, would be appreciated.
More clear initial communication of project course requirement in terms of reports and presentations (how many, which and when). Ideally a document should be generated which states all specifications to the project course (budget, reports, flight-options/rules, expected support from IRF and LTU, course rules for non-EU students, course registration if done in spring semester, possibility to order components from foreign countries).
Project budget wasn't tightly maintained by students in the beginning. But also, some costs (e.g. styrofoam and possibly more) were not communicated to the group and exact amount of budget per student was very unclear throughout most of the project.
Especially us spacemasters are quite restricted in our time schedule since many of us only have half a year in Kiruna, leaving around June and the rest leaving around January/February the following year. Hence, deadlines and time restrictions stated by these project groups should be well noted and respected.