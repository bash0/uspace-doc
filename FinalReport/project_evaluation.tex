\newpage
\chapter{Project Evaluation}
\label{chap:evaluation}
% provide feedback about the project course: what was good/fun/etc.?, what did you learn?, what didn't work well?, what can be improved for future projects?, other comments...
%This is mainly for feedback to Thomas, Kjell and LTU.

\section{Overall Project Goals}

Final budget status

What project goals were achieved?

Forward propulsion and steering and battery powered flight achieved along with feasibility of using solar power. No solar powered flight yet. Capability of system to support small scientific payloads (data handling, power supply, mechanical mounting etc.) No outdoors flight yet.

Verification of possibility to design a light-weight system with maximum weight 2.5 kg (including solar cells) which can be lifted by the Esrange blimp.

\section{Pedro}


\section{Omair}
It was an interesting and challenging project for me. I learnt different tools during the project phase. These include designing a communication protocol for a wireless system, programming of an arm processor, \ac{PCB} schematic design and especially their etching in the laboratory. Beside these technical tools, I also leant documentation of a technical project which will be very fruitful for me in future projects. I learn about the different parts of aircraft/spacecraft theoretically but \ac{U-SPACE} provided me the platform to get practical experience of about every subsystems of a satellite. In short, it is a good course which helps to work independently on in the area of personal interest. 

\section{Jan}


\section{Morten}
In overall, the project course was personally for me a useful experience. As a quality/project manager, I experienced many of the typical challenges in project management such as time delays, rephrasing of project goals and system concepts as required, bureaucracy issues, budget constraints, changes in team structure (people leaving, people joining), challenges getting access to components and equipment/labs etc. However, it is also my personal belief that many of these problems could easily have been avoided by better communication and organization with the LTU administration at the beginning of the project. This would allow more advanced (and complete) technical systems to be developed, focusing more on the real technical challenges leading to greater project successes and providing good PR source for LTU and the students when project results are displayed in media. It is my belief that LTU has some great facilities to support very interesting student projects and they should thrive to exploit and nurture this opportunity.
%
Some personal suggestions or comments for future projects:
\begin{itemize}
\item Establishment of Viking room as permanent project room.
\item Always access to a "professional" solder station, possibly even with ESD-protected table + wrist bands.
\item Throughout the project, we were extremely dependent on Lars to get simple components/tools (resistors, wires etc.). This cost us a lot of time, whenever he was not available. If possible, a box of standard components (resistors, capacitors, simple ICs, wires etc.) should be made available to at least one person (if not all) in the project team. A simple account-sheet could be placed next to the box, where people can note what they take.
\item Quicker release of initial project budget to allow first order of components.
\item Access to workshop, once established, worked very well (except some issues with the electronic lock). This access should be established from the beginning of the project.
\item Generally we received good support from IRF, borrowing tools and suggestions for mechanical design. If possible, more instructions and recommendations from IRF experts would be helpful (soldering course, how to check solderings under microscope).
\item More critical design feedback and suggestions from supervisors, where possible, would be good.
\item More clear initial communication of project course requirement in terms of reports and presentations (how many, which and when). Ideally a document should be generated which states all specifications to the project course (budget, reports, flight-options/rules, expected support from IRF and LTU, course rules for non-EU students, course registration if done in spring semester, possibility to order components from foreign countries, etc.).
\item Project budget wasn't tightly maintained by students in the beginning. But also, some costs (e.g. styrofoam and possibly more) were not communicated to the group and exact amount of budget per student was very unclear throughout most of the project.
\item Especially the Spacemaster students are quite restricted in their time schedule since many of us only have half a year in Kiruna, leaving around June and the rest leaving around January/February the following year. Hence, deadlines and time restrictions stated by these project groups should be well noted and respected.
\end{itemize}