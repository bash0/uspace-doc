\newpage
\chapter{Future Visions}
\label{chap:visions}
% future project visions for U-SPACE
%
It is the authors's believe that designing and building an airship is a great way for students to gain initial practical experiences on project work subjected to many of the same technical constraints also found in real space projects, including:
%
\begin{itemize}
\item Very limited power budget
\item Strong mass constraints due to limited lift mass
\item Thermal constraints considering daily and seasonal weather conditions, when flying outdoors
\item System autonomy, since no physical access to systems is possible once flying in the sky
\item Support requirements for scientific payloads
\item Mechanical structure optimized for low weight with strength and rigidity to withstand non-nominal situations (uncontrolled landings etc.)
\end{itemize}
%
Airships provide low cost and frequent flight opportunities thus allowing flight experience for all students involved in the project, even if only for a single semester.  Furthermore, if means are added to reuse helium gas after flight, one of the main expenses in these systems, flight costs can be reduced to almost zero.
Due to the recent popularity of private \ac{RC}-planes, many airship components can be found at low cost and high quality such as Li-ion batteries, motors, propellers and motor controllers.
\\\\
%
\noindent
The U-SPACE project has shown the way for many possible future upgrades which can be realized in student projects, including (but not limited to): 
%
\begin{itemize}
\item Realizing a completely solar powered flight in outdoor conditions.
\item Designing and building a custom envelope optimized for low weight and mechanical properties (rigidity, mounting options, ease of manufacturing etc.)
\item Altitude control system. This could be realized by having a compressor and an air-filled envelope. When inflating the air-filled envelope, it will compress the helium envelope thus reducing the lift of the airship (same concept as used on the old Zeppelins).
\item Attitude and flight path control system. Possibility to add coordinates and positions that the airship should automatically follow or maintain.
\item System up-scaling: Higher motor power, more lift capability to support larger and more sophisticated scientific payloads.
\item Scientific payloads: Atmospheric measurements (ice crystals, meteorological etc.), simple Synthetic Aperture Radar(as demonstrated by MIT lecturer in Optics and Radar course), space elevator demonstrator,...
\item Ground link with LTU ground station. Thus providing a professional high-bandwidth communication link and a test environment for students to gain experience as ground controllers who could then also support international cubesat missions around the globe.
\end{itemize}