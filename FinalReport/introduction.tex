\newpage
\chapter{Introduction}
\label{chap:introduction}
%
This document is the Final Report on the \ac{U-SPACE} project. This report describes major design changes implemented since the \ac{CDR}\cite{CDR} as well as the first obtained flight results. Project evaluations from each project team member is also included along with possible future work and recommendations.
\\\\
%
For a detailed description of the \ac{U-SPACE} project and system design, please refer to \cite{CDR}.
%
\section{Project Overview}
\ac{U-SPACE} is a student project at \ac{LTU} Rymdcampus in Kiruna and is supervised by Kjell Lundin, Thomas Kuhn and Alf Wikstr\"{o}m (now retired). The project is financially supported by \ac{LTU} and has also received some technical support from \ac{IRF} and Esrange Space Center. The goal of this project is to prove the concept of a small-scale student-built unmanned \ac{SPA}. The solar cells that power the airship are mounted on a gas-filled envelope, with forward propulsion being achieved by propellers mounted onto the same envelope. The airship communicates via two separate wireless connections using a remote control and a ground station. These communication channels enable human control over the airship, together with the retrieval of housekeeping and scientific payload data. The payload data consists of measurements from several sensors (magnetometer, accelerometer, gyroscope and \ac{GPS}) and images collected by a small on-board camera. On the ground station, the image data and the sensor data are combined to construct an aerial map of the flight environment.