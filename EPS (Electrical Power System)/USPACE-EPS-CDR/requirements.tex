\newpage
\section{Functional and Technical Requirements}
\label{sec:requirements}

\subsection{Functional Requirements}
%What function(s) does the subsystem have to fulfill?
%
Below are listed the primary functional requirements for the \ac{EPS}:
%
\begin{itemize}
\item Provide adequate power to motors and payload
\item Proof that flying on solar energy is possible i.e more power produced than consumed
\end{itemize}
%
Additional desired requirements are:
%
\begin{itemize}
\item Scalability to higher power levels
\item Flexible and robust design, allowing flight in more extreme conditions (altitude, weather etc.)
\item Provide adequate protection circuits for battery and loads
\item Optimal design and high performance to increase power capability and minimize system mass
\end{itemize}
%
%
\subsection{Technical Requirements}
%What technical requirements constrain the subsystem design? - e.g. mass, power, strength, stability etc.
%
The \ac{EPS} technical requirements are listed in table \ref{tab:technical_requirements}.
%
\begin{table}[H]
\centering
\caption{Technical requirements for the \ac{EPS}}
\label{tab:technical_requirements}
\begin{minipage}{\textwidth}
\begin{tabular}{p{0.4\textwidth}p{0.4\textwidth}}
\hline
Minimum power output & $40\,W$\\
Maximum mass & $1000\,g$(including solar arrays)\\
Maximum cost & $5000\,SEK$\footnote{Initial budget for 2 students.}\\
%Maximum internal power dissipation & $<500\,mW$\\
Output voltages & $6.0-9.2\,V$(un-regulated), $5\,V$( regulated)\\
Maximum output current (worst case) & $10.5\,A$\\
Regulator phase margin & $60\,deg$\\
Regulator gain margin & $10\,dB$\\
Control loop bandwidth & $>10\,kHz$\\
Operational temperature & $-20^{\circ}C\,to +25^{\circ}C$\\
Battery capacity & $>5\,Wh$\\
\hline
\end{tabular}\par
\vspace{-0.75\skip\footins}
\renewcommand{\footnoterule}{}
\end{minipage}
\end{table}
%
%\subsection{Mission and Environmental Constraints}
%\label{subsec:environmental_requirements}
%This section discusses some of the challenges opposed by the mission and the parameters of operation environment and how these will influence the \ac{EPS} design constraints.
%
%\subsubsection*{Solar Array Temperature}
%Temperature variation of the solar panels, significantly changes the solar panels characteristics and of main importance the location of the \ac{MPP}. In \cite{PDR} using the temperature coefficient of the open-circuit voltage of a proposed solar cell, the decrease in power output from the cell when the temperature goes from $-20^{\circ}C$ to $+25^{\circ}C$ was estimated to around $15-20\%$. The temperature characteristics of the chosen solar cells [reference] are not provided by the manufacturer, hence these should be tested in order to estimate the solar cell sensitivity to temperature changes. To mitigate this issue, a \ac{MPPT} is applied which will be explained in section [reference].
%
%
\subsection{Expected Performance}
%What are the expected performances of the subsystem, as related to the requirements above? (maybe including some margins)
%
\begin{table}[H]
\centering
\caption{Expected performance of the \ac{EPS}}
\label{tab:expected_performance}
\begin{minipage}{\textwidth}
\begin{tabular}{p{0.55\textwidth}p{0.35\textwidth}}
\hline
Power conversion efficiency(overall) & $80-90\%$\\
Power output(overall) & $\sim 57-65\,W$\\
Battery capacity & $7.3\,Wh$\\
Mass & $\sim910\,g$\\
Total cost & $\sim12000\,SEK$\footnote{Solar cells are significantly more expensive than anticipated. A request for more funds is under preparation.}\\
\hline
\end{tabular}\par
\vspace{-0.75\skip\footins}
\renewcommand{\footnoterule}{}
\end{minipage}
\end{table}