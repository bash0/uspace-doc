\newpage
\chapter{Introduction}
\label{chap:introduction}

\ac{U-SPACE} is a student project at the Rymdcampus of the \ac{LTU} in Kiruna under the supervision of Kjell Lundin and Alf Wikström. It is supported by the \ac{IRF} and \ac{LTU}. The goal of the project is to prove the concept of a small scale student-built unmanned \ac{SPA} powered by solar cells. The solar cells are mounted on a gas-filled envelope, with forward propulsion being achieved by propellers mounted on the same envelope. The airship communicates over a wireless connection with a ground station or controller. This connection enables  control over the airship, together with retrieval of housekeeping and scientific payload data.
\\
\\
The same concept of a \ac{SPA} has attracted major interest in recent years \cite{website:ravenaerostar, website:gaya, poster:saba, report:colozza2004}. Such an airship could be used for a wide variety of applications, ranging from passenger and cargo transport \cite{website:gaya} over scientific research \cite{poster:saba} to planetary exploration \cite{report:colozza2004}. These applications all benefit greatly from the advantages of a solar-powered airship: simple flight control, reduced fossil fuel consumption and access to long duration flights. Apart from these inherent strong points, other advantages of \ac{SPA}s are the possibility for autonomous take-off and landing, the elimination of large infrastructures like airports and minimal weather constraints. Even though many researchers have investigated the possibilities of \ac{SPA}'s, few student-driven projects exist. \textbf{EXAMPLES OF STUDENT PROJECTS?} The above-mentioned advantages of \ac{SPA}'s and the fact that few student projects exist, were the main drivers for the creation of the \ac{U-SPACE} project.

\section{Hardware}
\label{sec:intro_hardware}

Description of the hardware, including a block diagram

\section{Software}
\label{sec:intro_software}

Description of software, including operational modes