\chapter{Mechanical Structure and Envelope}
\label{chap:mse}

%General introduction to the subsystem...

\section{Functional and Technical Requirements}

\textit{Will be added in the final version...}

\section{Mechanical and Structural Design}

As was mentioned in the \ac{PDR}, the initial mechanical design of the \ac{U-SPACE} project included the development of a blimp (envelope) that would then accommodate the solar panels of the \ac{EPS}, the cargo bay for the \ac{ITPU} and the propelling system of \ac{MCC}. An initial sketch of this concept is presented in figure \ref{fig:init}.  

\begin{figure}[bht]
\centering
\includegraphics[scale=0.5]{figures/init.png}
\caption{Initial 3D sketch of the airship}
\label{fig:init}
\end{figure}

This simplified view of the mechanical design shows that it included the total development of the envelope of the airship. The idea was to include the solar panels on the top, mounted on a wired mesh, with the cargo bay attached in the bottom together with the propelling system.
\\
\\
However, due to new developments in the project that included the introduction of an already built blimp (envelope), the focus of the mechanical design changed. The blimp to be used would be the TIF - 250 (Tethered Aerostats), shown in figure \ref{fig:blimp}. This blimp has the capacity to lift a payload of about 2.5 kg, has a length of approximately 5 m and has a diameter (in the center) of about 1.9 m.

\begin{figure}[bht]
\centering
\includegraphics[width=\textwidth]{figures/blimp.jpg}
\caption{TIF - 250 Blimp}
\label{fig:blimp}
\end{figure}



This blimp serves the purpose of the U-SPACE project very well, as its use would allow to focus only on the construction of the support for the power system, the cargo bay (for the payload) and the integration of the propelling system. However, because it is a ready-built blimp with a purpose different than the one envisioned, it is not as lightweight as might be needed. Nevertheless, an effort will be made to include light structures in the integration of all the other systems in this airship.

\subsection{Envelope}

As it was already stated, the blimp to be used is a ready-built one. This blimp is normally used to accurately measure the wind direction. Nevertheless it will have to fit the purpose of the U-SPACE project, due to the lack of time to build a new envelope. This way, the envelope is constituted by the blimp itself.  

\subsection{Cargo Bay}

The cargo bay is intended to accommodate both the payload and the electronics required for the purpose of the project. The main challenge in the construction of this cargo bay is the weight. It has to be lightweight (the total maximum weight of all the structures should be less than 1 kg) but at the same time rigid enough to resist some stress during the normal operation of the airship. To achieve this, balsa wood reinforced with carbon fibres was used. Figures of the expected final cargo bay design and of the current construction status are presented in figures \ref{fig:box} and \ref{fig:boxinit}, respectively. 

\begin{figure}[bht]
\centering
\includegraphics[scale=0.5]{figures/box.png}
\caption{3D sketch of the cargo bay} 
\label{fig:box}
\end{figure}

\begin{figure}[bht]
\centering
\includegraphics[width=\textwidth]{figures/boxinit.jpg}
\caption{Initial construction phase of the cargo bay}
\label{fig:boxinit}
\end{figure}

\subsection{Power System}

The biggest challenge of this project is to accommodate the power system, taking into account the maximum lift weight and also the power requirements that consequently influence the solar panel quantity and weight. Because different solar panels are still under test, it is still not decided how they will be mounted on the blimp. Nevertheless, the idea is to use a lightweight wired mesh that serves as a support to the solar panels, which are attached to the wires with carbon fibre. This mesh will then be connected to the blimp making use of 3 bands that will round the blimp, distributing the weight along the envelope. These bands will be made of fibre glass reinforced rubber tape. An idea of how the final product should look like is presented in figure \ref{fig:mesh}.

\begin{figure}[bht]
\centering
\includegraphics[scale=0.3]{figures/mesh.jpg}
\caption{3D sketch of the integration of the power system}
\label{fig:mesh}
\end{figure}

\subsection{Propelling System}

The propelling system is to be integrated in a carbon fibre rod mounted on the top of the cargo bay. The 2 motors will be attached at the ends of the rod, outside the influence of the envelope and free to achieve their maximum aerodynamic capabilities. A hand sketch of this principle is showed in figure \ref{fig:prop}.

\begin{figure}[bht]
\centering
\includegraphics[width=\textwidth]{figures/prop.jpg}
\caption{Hand sketch of the integration of the propelling system}
\label{fig:prop}
\end{figure}

\section{Future Developments}

All the previously explained designs have to be built and tested. Conclusions have to be made and inputs from the other subsystems have to be taken into consideration. Only after a careful building of the different structures to accommodate all the required subsystems, it will be possible to check if the requirement of the maximum lift weight - the most important requirement - is achieved. For now, the 3D designs, the envisioned materials and previous experiences in the field give hope that this constraint will be surpassed.
\\
\\
The following steps should be to finish the construction of the cargo bay, accommodate the propelling system into it and then proceed with the construction of the wiring mesh and the consequent attachment of the solar panels.
\\
\\
\textit{Topics such as physical properties, structural and mechanisms analysis and mounting attachments will be discussed in the final version...}
%\section{Mechanical Interfaces}
%
%How does the MSE system interact with the other subsystems?
%
%\subsection{Mechanical Interface Control Drawing}
%
%Could not really find what this is and if we need it... Morten?
%
%\subsection{Accommodation Requirements}
%
%Same here...
%
%\section{Physical Properties}
%
%E.g. mass in launch configuration...
%
%\section{Structural and Mechanisms Analysis}
%
%This involves things like dynamic analysis and stress analysis, but as we didn't really do this, just briefly comment on it...
%
%\section{Mounting Attachments}
%
%Not sure what they mean with this... Attachment concept and foot pattern?