\chapter{Thermal Interfaces, Pyrotechnics and Electromagnetic Compatibility}
\label{chap:thermal_pyro_emc}

\section{Thermal Interfaces}

The topic of thermal interfaces is irrelevant for the \ac{U-SPACE} project. All subsystems, both for the airborne structure and the ground-based structures, will have roughly the same temperature during the entire operation period of the \ac{SPA}. Also the expected "omgevingstemperatuur" is uniform and not extremely high or low (see section \ref{sec:constraints} in chapter \ref{chap:goals_constraints}). For these reasons it is not necessary to investigate the issues related to thermal interfaces (e.g. thermal mismatch).

\section{Pyrotechnics Interface}

In order to achieve the functionality described in chapter \ref{chap:goals_constraints} there is no need for any form of pyrotechnics. The project does not require the release or deployment of other structures which could make use of pyrotechnics. Therefore pyrotechnics particulars are not discussed in this respect.

\section{Electromagnetic Compatibility}

With regards to \ac{EMC}, no problems are expected during the \ac{U-SPACE} project. The frequencies employed in the subsystems are compatible with each other and with the regulations from the \ac{ITU} \cite{book:freqalloc}. More information regarding the selection of the relevant frequencies and their use in the \ac{U-SPACE} project can be found in chapters \ref{chap:mcc} and \ref{chap:itpu}.
\\
\\
The possible issues related to grounding are briefly discussed in chapter \ref{chap:eps}. Once again the requirements are not very stringent, meaning that few issues are expected related to this topic.