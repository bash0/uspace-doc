\chapter{Thermal Interfaces, Pyrotechnics and Electromagnetic Compatibility}
\label{chap:thermal_pyro_emc}

\section{Thermal Interfaces}
This section describes the \ac{U-SPACE} thermal requirements and design.
%
\subsection{Thermal Requirements}
The \ac{U-SPACE} systems are required to operate in the temperature range $-20^{\circ}\,C$ to $+45^{\circ}\,C$. The limits are based on weather data from Table \ref{tab:environment} including appropriate design margins. The high temperature limit also considers heat contribution from direct external sun exposure and internal power dissipation from the \ac{EPS}. Table \ref{tab:temp_critical_parts} lists temperature critical parts that are not rated for the full temperature range as given above.
%
\begin{table}[H]
\centering
\caption{Temperature critical parts}
\label{tab:temp_critical_parts}
\begin{minipage}{\textwidth}
\begin{tabular}{p{0.2\textwidth}p{0.2\textwidth}p{0.15\textwidth}p{0.15\textwidth}p{0.2\textwidth}}
\hline
\textbf{Part} & \textbf{Rating} & \textbf{Monitoring} & \textbf{Control} & \textbf{Protection}\\
\hline
BeagleBoard \ac{MCU} & $0^{\circ}\,C$ to $+85^{\circ}\,C$ & - & - & ?\\
\hline
\ac{EPS} Battery &  $+5^{\circ}\,C$ to $+40^{\circ}\,C$\footnote{Including $5^{\circ}\,C$ design safety margin} & External thermistor & - & Charge inhibit at low temperature\\
\hline
Current sense OpAmp & $0^{\circ}\,C$ to $+70^{\circ}\,C$ & - & - & -\\
\hline
\end{tabular}\par
\vspace{-0.75\skip\footins}
\renewcommand{\footnoterule}{}
\end{minipage}
\end{table}

\subsection{Thermal Design}
From Table \ref{tab:temp_critical_parts} it is mainly the lower temperature limit which is of concern. Currently the \ac{U-SPACE} thermal design does not include temperature monitoring, except the battery, nor any temperature control. It will therefore not be possible to fly when the outdoor temperature falls much below $+5^{\circ}\,C$. This issue may be solved by insulating the cargo bay with Styrofoam or similar material. Internal dissipation from the \ac{EPS} and payloads will increase the internal cargo bay temperature. Additional heaters may also be included in the design, however this will complicate the system and consume power, which is already limited.\\[5mm]
Complete thermal analysis, insulation requirements and possible heater design still remains to be done.
%
\section{Pyrotechnics Interface}

In order to achieve the functionality described in chapter \ref{chap:goals_constraints} there is no need for any form of pyrotechnics. The project does not require the release or deployment of other structures which could make use of pyrotechnics. Therefore pyrotechnics particulars are not discussed in this respect.

\section{Electromagnetic Compatibility}

With regards to \ac{EMC}, no problems are expected during the \ac{U-SPACE} project. The frequencies employed in the subsystems are compatible with each other and with the regulations from the \ac{ITU} \cite{book:freqalloc}. More information regarding the selection of the relevant frequencies and their use in the \ac{U-SPACE} project can be found in chapters \ref{chap:mcc} and \ref{chap:itpu}.
\\
\\
The possible issues related to grounding are briefly discussed in chapter \ref{chap:eps}. Once again the requirements are not very stringent, meaning that few issues are expected related to this topic.