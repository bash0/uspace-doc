\chapter{Project Management}
\label{chap:project_management}

\section{Organisation and Responsibilities}

A project of the extent of the \ac{U-SPACE} project, even though it is a student project, requires clear agreements between the team members with with respect to responsibilities, work load and communication. Also the available support facilities and other practical issues have to be investigated beforehand.

\subsection{Key Personnel and Responsibilities}

Two important team members in the \ac{U-SPACE} project are the quality manager and the project manager. These managers were elected by the entire team. The quality manager, Morten Olsen, takes care of the technical side of the project. He is responsible for topics such as technical consistency between the different subsystems, follow-up of the progress of each subsystem and finally for guaranteeing the general technical quality of the project. The quality manager therefore gathers updated information from each subsystem, synthesizes this information and informs the project manager when problems arise. Currently Morten Olsen has also taken over the responsibilities of Dries Agten as project manager of the \ac{U-SPACE} project.
\\
\\
The project manager, Dries Agten, has responsibilities ranging from practical organisation over general communication to project documentation. The practical organisation includes the call for a weekly meeting, chairing this meeting, checking meeting reports, etc. With regards to communication the project manager is responsible both for internal and external communication. Therefore the project manager is in contact with the supervisors, Kjell Lundin (\ac{IRF}) and Alf Wikström (\ac{LTU}), to communicate the progress of the project and to have regular meetings to discuss this progress.  The internal communication topics range from calls for meetings to the passing-on of important information gained from the contact with the supervisors. Taking care of the project documentation means that the project manager is responsible for the design reviews, the meeting reports and other documentation that may be generated over the course of the project.

\subsection{Functional Organigram}

A functional organigram presenting the different subsystems (as defined in chapter \ref{chap:introduction}) and the associated team members is shown in figure ref{fig:obs}. This organigram is based on the background and expertise of the different team members (see table \ref{tab:backgrounds}), such that each team member is assigned a responsibility which he is able to handle.

\begin{figure}[bht]
\centering
\includegraphics[width = \textwidth]{figures/obs.png} %scale=0.5
\caption{Functional organigram}
\label{fig:obs}
\end{figure}

\begin{table}[h]
\centering
\caption{Background of team members}
\begin{tabular}{l l}
\hline
\textbf{Name} & \textbf{Background} \\
\hline
Dries Agten & M.Sc. Eng. - Nanoscience \& Nanotechnology \\
Mauro Aja & B.Sc. Eng. - Electronic and Computer Engineering \\
Ishan Basyal & B.Sc. Earth and Space Sciences  \\
Pedro Cervantes & B.Sc. Eng. - Aeronautical Engineering \\
Bastian Hacker & B.Sc. Physics\\
Daniel Zhou Hao & B.Sc. Eng. - Aerospace Engineering\\
Morten Olsen & B.Sc. Eng. - Electrical Engineering \\
Oliver Porges & B.Sc. Cybernetics and Measurement  \\
Anuraj Rajendraprakash & B.Tech. Electronics Engineering \\
Tiago Rebelo & B.Sc. Eng. - Aeronautical Engineering \\
Jan Sommer & B.Sc. Computational Science \\
\hline
\end{tabular}
\label{tab:backgrounds}
\end{table}

\subsection{Support Facilities}

The support facilities of the \ac{U-SPACE} project are the responsibility of three organisations, each with a focus on a specific part of the project. The first organisation is \ac{LTU}, represented by Alf Wikström. As all team members are students from \ac{LTU}, this university is the principal support facility during the \ac{U-SPACE} project. Its responsibility is the practical, financial and also technical support of the team members. The practical support consists of providing tools and specific work spaces, such as mechanical workshops or electronics laboratories. Some other aspects are discussed in more detail in section \ref{sec:relation_support}.
\\
\\
The two remaining organisations are \ac{IRF} and Esrange. \ac{IRF} is represented by Kjell Lundin, while the contacts with Esrange are the responsibility of Alf Wikström. These two support facilities provide technical assistance during the course of the project, consisting of guidance regarding the selection of the components, support during the fabrication procedures and recommendations regarding the test set-up.

\subsection{Shipment}

The shipment of the prototype is an important topic in this project. Since the prototype will be built at the Rymdcampus of \ac{LTU} and since the test flight will most probably take place at Esrange, a secure transport procedure has to be developed. As the envelope is the property of Esrange (see chapter \ref{chap:mse}), the final assembly of the airship can only take place at this location, which relieves the transport constraints. Each subsystem has a limited size which can be  transported easily in a normal passenger car. When the final assembly is completed, this mode of transport is no longer possible. In case the prototype needs to be transported after assembly a small truck will be required in order not to damage the airship.

\section{Relation With Support Facilities}
\label{sec:relation_support}

The support facilities mentioned in the previous section play an important role in the course of the \ac{U-SPACE} project. Their responsibilities are wide, ranging from reports over components to finances.

\subsection{Reporting and Monitoring}

During the entire project the team is monitored by the supervisors Kjell Lundin (\ac{IRF}) and Alf Wikström (\ac{LTU}). This monitoring task consists of controlling the budget, approving the ordering of components and evaluating the (technical) progress of the team at strategical points in time. Weekly meetings between the two supervisors and the two team managers are scheduled to identify, discuss and solve all issues that have come up since the last meeting. Apart from these weekly meetings, communication via email and telephone is used to keep the supervisors updated about the status of the project and to ask for assistance when needed.

\subsection{Reviews}

Apart from the weekly meetings discussed above, two essential evaluation moments during the course of the project are the design reviews. These reviews are also the points in time when grades are connected to the work executed up to those points. The first design review is the \ac{PDR}, which occurred in early April. With the help of five separate documents, the general project and the different subsystems were introduced to the supervisors, with a focus on the preliminary designs. This \ac{PDR} marked the point of approval by the supervisors, after which further design and construction steps could be undertaken.
\\
\\
The second and final design review is the \ac{CDR}. In this document, which is the current document, the final designs of the subsystems are presented, together with some early construction results. It was accompanied by an oral team presentation in the middle of June.

\subsection{Component Ordering}

The ordering of the components is the responsibility of \ac{LTU}, the organization that is also responsible for the project financing (see section \ref{sec:financing}). The members of the subsystems select the appropriate components and pass the order on to the project manager. He consults with the project supervisors and after approval the components are ordered. Components can both be ordered from inside Sweden and from abroad.

\section{Financing}
\label{sec:financing}

The \ac{U-SPACE} project is financed by \ac{LTU}, the principal support facility. For each European team member, the team receives 2,000 SEK, amounting for a total budget of 14,000 SEK.This budget has to be used for the ordering of all components and for any other expenses that might arise during the course of the project (e.g. tools). An increase in the budget can be realized after an application procedure and approval by \ac{LTU}. In table \ref{tab:financing} all expenses and incomes of the \ac{U-SPACE} project are summarized.

\begin{table}[H]
\centering
\caption{Expenses and incomes}
\label{tab:financing}
\begin{tabular}{c c c}
\hline
\textbf{/} & \textbf{/} & \textbf{/}\\ \hline
/ & / &\\
\hline
\end{tabular}
\end{table}

\textit{I do not have any clear picture on the current budget, so if you can give me an update, Morten, I will be happy to include this table in the CDR :-)}

\section{Schedule and Milestones}

Since the \ac{U-SPACE} project is a student project that has to be realized within a limited amount of time a tight schedule and a clear definition of important milestones are vital for the success of the project. The schedule has to updated regularly to reflect the current progress and/or delays of the project. The most recent schedule is shown in figure \ref{fig:schedule} below. All important milestones are indicated as well.

\begin{figure}[htbp!]
\centering
\includegraphics[width=\textwidth]{figures/schedule.png}
\caption{Current schedule and milestones.}
\label{fig:schedule}
\end{figure}

\textit{Morten, since I do not really have an idea about the status, I was hoping that you might be able to add this figure to the text. You know better what future milestones have to be realized as well...}

\section{Configuration Control}

Although the \ac{U-SPACE} project is limited in size, configuration control remains an important element of the success of the project. With four different subsystems and many different team members it is essential to have a common platform where design changes are listed, tracked and discussed. This is necessary to guarantee technical consistency and to avoid misunderstandings that may endanger the outcome of the project.
\\
\\
The main component of the \ac{U-SPACE} configuration control is a weekly meeting of all team members. An agenda is composed beforehand by the meeting chair, listing all important issues, both practical and technical, that have to be discussed during the meeting. A meeting secretary is responsible for producing minutes of the meeting. These minutes, together with the agendas, are accessible for the entire team by the use of Google Docs. Another minor component of the configuration control is the revision control system GitHub, which is mainly used to share the documents that make up the design review. It is also the place where the developed \ac{ITPU} code is made available to the relevant team members. For minor issues, emails to a common team email address are used.

\section{Deliverables}

The ultimate goal of the \ac{U-SPACE} project is to realize a functioning prototype of a \ac{SPA} capable of forward propulsion while supporting a scientific payload. Since this project is a student project under the supervision of \ac{LTU}, the final phases of the project also require the handing in of several deliverables.

\subsection{Hardware and Software}

The main deliverable is a functioning prototype of an \ac{SPA}, consisting of several separately developed subsystems (see chapters \ref{chap:mse} through \ref{chap:itpu}). Each subsystem should function on its own as well as in concurrence with the other subsystems. All necessary components have to be included when the subsystem is delivered.
\\
\\
Also the ground station is a deliverable, although it is mainly software-based. All code has to be delivered at the end of the project, together with comments that explain the code (see the following subsection). Also the code written for the airborne part of the \ac{ITPU} has to be included.

\subsection{Documentation}

The deliverable documentation first of all consists of both design reviews (\ac{PDR} and \ac{CDR}), which are documents that provide details on the functioning of each subsystem and on the general elements of the project. When available, all documents that were used during the development of a subsystem should be classified and stored for future use. This allows future team members to easily pick up the work where it was left by their predecessors.
\\
\\
Apart from these hardware-related documents, documentation regarding the developed software should also be delivered. This documentation mainly consists of comments on the code such that it may be understood and further developed by future team members.