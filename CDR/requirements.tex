\chapter{Goals and Constraints}
\label{chap:goals_constraints}

This chapter lists the goals and constraints of the \ac{U-SPACE} project. %Section \ref{sec:functional} contains the technical requirements, while section \ref{sec:technical} summarizes the technical requirements. 

\section{Project Goals}
%What function(s) does the subsystem have to fulfill?

The goal of the U-SPACE project is to design, build and test a \ac{SPA} for low altitude flying while supporting a scientific payload. Two future goals are implementing autonomous attitude control and altitude control of the airship.

\section{Project Constraints}
%What technical requirements constrain the subsystem design? - e.g. mass, power, strength, stability etc.

The constraints that have to be taken into account for this project can be divided into four categories: resources, functionality, environment and law. With regards to the resources, the following constraints can be identified:

\begin{itemize}
\item Limited time of one month and a half (part-time project work)
\item Limited budget (10,000-12,500 SEK)
\item Limited expertise in the field of airships
\end{itemize}

\noindent
Two functional/technical constraints that have to be taken into account are:

\begin{itemize}
\item Electrical power is limited to 8 W.
\item Mass is limited to 4.5 kg.
\end{itemize}

\noindent
As this project will be executed during the months of April, May and June in the city of Kiruna, the environmental conditions are as listed below (using data for the month of May \cite{website:weatherspark}):

\begin{itemize}
\item Average temperatures between -2 $^\circ$C and 11 $^\circ$C 
\item Typical wind speeds ranging from 1 m/s to 6 m/s (usually coming from the south west)
\item Median cloud cover of 83 \%
\item Average probability of precipitation of 61 \%
\item Daily hours of sunshine between 17:38 hours and 22:45 hours
\item Solar incidence angle of 46 $^\circ$ (refer to document USPACE-PDR-PWR-A1)
\end{itemize}

\noindent
Finally, two legal issues with regards to the project are presented below:

\begin{itemize}
\item The \ac{ITU} Radio Regulations \cite{book:freqalloc} might have to be taken into consideration.
\item The Swedish Transport Agency's regulations on \ac{UAS} \cite{regulations:uas2009} might have to be taken into account.
\end{itemize}

\section{Expected Results}
%What are the expected performances of the subsystem, as related to the requirements above? (maybe including some margins)

Referring to the project goals and constraints mentioned above, the expected result of the project is a small scale prototype of a \ac{SPA} capable of:

\begin{itemize}
\item operating autonomously for 2 hours at peak power consumption,
\item flying at an altitude between 1 m and 20 m,
\item flying with a forward velocity between 0.5 m/s and 1 m/s,
\item flying during daytime in sunny and calm weather conditions,
\item supporting a scientific payload during its entire operating time.
\end{itemize}

\section{Fault Tolerance Design, Safety Concept and Materials}

\textit{Will be added in the final version...}

%\section{Fault Tolerance Design and Safety Concept}
%
%N/A, but comment on it. Refer to MGSE for safety concept
%
%\section{Materials}
%
%Briefly comment on it

%\section{Functional Requirements}
%\label{sec:functional}
%
%What function(s) does the system have to fulfill?
%
%\begin{itemize}
%\item A requirement
%\item Another requirement
%\item Etc...
%\end{itemize}
%
%
%\section{Technical Requirements}
%\label{sec:technical}
%
%What technical requirements constrain the system design? - e.g. mass, power, strength, stability etc.
%
%\begin{itemize}
%\item A requirement
%\item Another requirement
%\item Etc...
%\end{itemize}