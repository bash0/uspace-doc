\chapter{Electromagnetic Compatibility}
\label{chap:emc}
%
This chapter describes the \ac{U-SPACE} \ac{EMC} issues, the \ac{EMI}-susceptible parts and the mitigation methods. For the \ac{U-SPACE} \ac{EMC} one mission-critical circuit has been identified. This is the motor control of the \ac{MCC} subsystem. Loss of motor control due to \ac{EMC} issues could in the extreme case result in equipment damage. Two non-critical circuits have been identified as well: the \ac{ITPU} sensors and the telemetry communication system. 
%
\section{Electromagnetic Interference Sources}

The identified main \ac{EMI} sources are:
%
\begin{itemize}
\item The \ac{EPS} \ac{SAR}: radiated \ac{EMI} from the power inductor and from high AC-current loops. Also conducted \ac{EMI} from the power switches
\item The two \ac{MCC} DC-motors: a DC magnetic field (H-field) from the permanent magnet rotor and radiated \ac{EMI} when the motors are running
\item The \ac{MCC} motor power leads: radiated \ac{EMI} from large AC-current loops (up to $\sim 7.5\,A$ per motor).
\end{itemize}
%
\section{Electromagnetic Interference Susceptibility}
%
Table \ref{tab:EMI_susceptibility_mitigation} lists the \ac{EMI}-sensitive parts which have been identified along with the applied mitigation method.
%
%
\begin{table}[H]
\centering
\caption{EMI-susceptible parts and mitigation methods}
\label{tab:EMI_susceptibility_mitigation}
%\begin{minipage}{\textwidth}
\begin{tabular}{p{0.25\textwidth}p{0.25\textwidth}p{0.4\textwidth}}
\hline
\textbf{Susceptible part} & \textbf{Susceptibility type} & \textbf{Mitigation method}\\
\hline
\ac{ITPU} magnetometer & DC H-fields &	Must be kept at minimum distance from DC-motors. Calibration might be necessary after \ac{ITPU} integration in complete \ac{U-SPACE} system.\\
\hline
\ac{ITPU} sensors & \rr Radiated and conducted \ac{EMI} & Must be kept at minimum distance from \ac{EMI} radiation sources.\tn
\hline
\ac{MCC} motor communication & Radiated \ac{EMI} & Must be kept at minimum distance from \ac{EMI} radiation sources.\\
\hline
\rr \ac{MCC} telemetry communication & Radiated \ac{EMI} & Must be kept at minimum distance from \ac{EMI} radiation sources.\tn
\hline
\end{tabular}%\par
%\vspace{-0.75\skip\footins}
%\renewcommand{\footnoterule}{}
%\end{minipage}
\end{table}
%
%
\section{Electromagnetic Interference Mitigation Methods}
In the \ac{U-SPACE} project, the primary applied method for reducing \ac{EMI}-related issues is to separate \ac{EMI} sources from \ac{EMI}-susceptible parts by appropriate distances. It is also important to minimize the area of high AC-current loops. This can be achieved by a proper \ac{PCB} layout of the \ac{SAR} and by keeping together and twisting the forward and return power leads of the DC-motors. A secondary method of mitigating radiated \ac{EMI} from the \ac{SAR} will be to apply shielding either around the complete circuit or only around the power inductor. This will however increase the system mass and is therefore only a last-resort option.
%
%
\section{Electromagnetic Interference Tests}
\ac{EMI} tests for the \ac{EPS} have been defined in section \ref{sec:eps_test_program}. In general, susceptibility tests should be applied to the sensible parts defined in Table \ref{tab:EMI_susceptibility_mitigation}. When the \ac{EPS} \ac{SAR} and DC-motors are running at full power, no performance/communication degradation or glitches should be noticed due to \ac{EMI}. Guides for \ac{EMI} emission measurements and susceptibility tests along with accepted limits are provided in \cite{ECSS_EMC}.