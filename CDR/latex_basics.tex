\chapter{Basic LaTex Commands}
This section provides some basic useful LaTex commands. For further reference, search on Google where you will find plenty of useful LaTex blogs. \textbf{Remove this chapter later on...}

\section{Figures}

This is a figure example:

\begin{figure}[H] % the "H" specifies to place the figure exactly here in the document - If left out, LaTex will sometimes place figures a bit awkwardly.
\centering
\includegraphics[scale=1]{Drawing1} % it is recommended to use pdf or another vector graphic file for optimal quality
\caption{This is a figure caption}
\label{fig:figure1_label}
\end{figure}

You can also place figures side-by-side. An easy way is to use a "minipage" environment:

\begin{minipage}[b]{0.45\linewidth}  % "b" means that the bottom-line of the "minipages" will be aligned
\begin{figure}[H]
\centering
\includegraphics[width=0.9\textwidth]{Drawing1}
\caption{This is a figure caption}
\label{fig:figure2_label}
\end{figure}
\end{minipage}
\hspace{2mm} % include some horizontal space between the two figures
\begin{minipage}[b]{0.45\linewidth}
\begin{figure}[H]
\centering
\includegraphics[width=0.7\textwidth]{Drawing1}
\caption{This is a figure caption}
\label{fig:figure3_label}
\end{figure}
\end{minipage}

An alternative is to use the "subfigure" command inside the "figure" environment. You need the
\begin{verbatim}
\usepackage[center]{subfigure}
\end{verbatim}
command in your preamble. With this command, the figures will be labelled a, b, c etc.

\begin{figure}[htbp!]
\centering
\subfigure [Caption for subfigure 1]{\label{fig:figure4_label}\includegraphics[width=0.49\textwidth]{Drawing1}}
\subfigure [Caption for subfigure 2]{\label{fig:figure5_label}\includegraphics[width=0.49\textwidth]{Drawing1}}
\caption{General caption} 
\label{fig:figure4_5_label}
\end{figure}

\section{Tables}

This is an example of a table:

\begin{table}[H]
\centering
\caption{This is a table caption}
\label{tab:table1_label}
\begin{tabular}{|l|l|l|} % "|" means a border, "l" means left-aligned text in cells.
\hline % adds a horizontal border
\textbf{Header 1} & \textbf{Header 2} & \textbf{Header 3}\\ %use "&" to separate cells and remember a "\\" linebreak in the end
\hline 
Some text & Some text & Some text\\
Some more text & Some more text & Some more text\\
\hline
\end{tabular}
\end{table}

You can also do a table with multi-line cells:

\begin{table}[H]
\centering
\caption{This is a table caption}
\label{tab:table2_label}
\begin{tabular}{|p{0.4\textwidth}p{0.3\textwidth}p{0.2\textwidth}|}
\hline
\textbf{Header 1} &  \textbf{Header 2} & \textbf{Header 3}\\ 
\hline
Some long text that does not fit in a single-line table cell & Some text & Some text \\
Some more text & Another very long text that does not fit in a single-line table cell & Some more text\\
\hline
\end{tabular}
\rowcolors{3}{tableshade}{white}	% adds alternating background color of the table rows
\end{table}

\section{Equations}

You can do simple in-line equations by using the "\$" symbols around the equation: $2+2=4$. Remember always to use a the math- or equation environment when using variables like $+$, $=$, $x^{2}$, $f_{2}$ etc.

To write a numbered equation on its own line, use the "equation" environment: 

\begin{equation}
\label{eq:equation1_label}
T(s)=\frac{G(s)H(s)}{1+G(s)H(s)}
\end{equation}

You can also do multi-line equation by using the "split" - environment:

\begin{equation}
\begin{split}
\label{eq:equation2_label}
2x+4y&=6\\	%use "&" to align the equations over the same point
4y&=6-2x\\
y&=1.5-0.5x\\
\end{split}
\end{equation}

\section{Citations, References and Acronyms}

This is a citation\cite{CitationReference1}.

This is a citation referring to a specific page in the cited work\cite[28]{CitationReference1}. 
\\
You can also do multiple citations\cite{CitationReference1,CitationReference2}.

This is a cross-reference to a figure/section/table/equation etc. in the latex document: see Figure \ref{fig:figure1_label}.
\\
Use acronyms consistently to provide an easy-reading text: The \ac{U-SPACE} project rocks!